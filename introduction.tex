\chapter{Introduction}
\chapterprecis{Restructured, reorganized, and parts rewritten by Jim Mock}

\section{Synopsis}

Thank you for your interest in FreeBSD! The following chapter covers various
aspects of the FreeBSD Project, such as its history, goals, development model,
and so on.

After reading this chapter you will know
\begin{itemize}
\item
   how FreeBSD relates to other computer operating systems,
\item
   the history of the FreeBSD Project,
\item
   the goals of the FreeBSD Project,
\item
   the basics of the FreeBSD open-source development model, and, of course,
\item
   where the name ``FreeBSD'' comes from.
\end{itemize}



\section{Welcome to FreeBSD!}

FreeBSD is an Open Source, standards-compliant Unix-like operating system for
x86 (both 32 and 64 bit), ARM®, AArch64, RISC-V®, MIPS®, POWER®, PowerPC®, and
Sun UltraSPARC® computers.
It provides all the features that are nowadays taken for granted, such as
preemptive multitasking, memory protection, virtual memory, multi-user
facilities, SMP support, all the open source development tools for different
languages and frameworks, and desktop features centered around X Window System,
KDE, or GNOME.
Its particular strengths are:
\begin{itemize}
\item
   a liberal open source license, which grants you rights to freely modify and
   extend its source code and incorporate it in both open source projects and
   closed products without imposing restrictions typical to copyleft licenses,
   as well as avoiding potential license incompatibility problems.
\item
   strong TCP/IP networking---FreeBSD implements industry standard protocols
   with ever increasing performance and scalability.
   This makes it a good match in both server, and routing/firewalling roles---%
   and indeed many companies and vendors use it precisely for that purpose.
\item
   fully integrated OpenZFS support, including root-on-ZFS, ZFS boot
   environments, fault management, administrative delegation, support for jails,
   FreeBSD specific documentation, and system installer support.
\item
   extensive security features, from the mandatory access control framework to
   Capsicum capability and sandbox mechanisms.
\item
   over 30 thousand prebuilt packages for all supported architectures, and the
   Ports Collection which makes it easy to build your own, customized ones.
\item
   documentation---in addition to this handbook and books from different authors
   that cover topics ranging from system administration to kernel internals,
   there are also the \man[1]{man} pages, not only for userspace daemons,
   utilities, and configuration files, but also for kernel driver APIs
   (section~9) and individual drivers (section~4).
\item
   simple and consistent repository structure and build system---FreeBSD uses a
   single repository for all of its components, both kernel and userspace.
   This, along with an unified and easy to customize build system and a well
   thought out development process makes it easy to integrate FreeBSD with build
   infrastructure for your own product.
\item
   staying true to Unix philosophy, preferring composability instead of
   monolithic ``all in one'' daemons with hardcoded behavior.
\item
   binary compatibility with Linux, which makes it possible to run many Linux
   binaries without the need for virtualisation.
\end{itemize}

FreeBSD is based on the 4.4BSD-Lite release from Computer Systems Research Group
(CSRG) at the University of California at Berkeley, and carries on the
distinguished tradition of BSD systems development.
In addition to the fine work provided by CSRG, the FreeBSD Project has put in
many thousands of man-hours into extending the functionality and fine-tuning the
system for maximum performance and reliability in real-life load situations.
FreeBSD offers performance and reliability on par with other open source and
commercial offerings, combined with cutting-edge features not available anywhere
else.

\subsection{What Can FreeBSD Do?}