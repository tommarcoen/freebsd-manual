\chapter{Preface}

\section{Intended Audience}

The FreeBSD newcomer will find that the first section of this book guides the
user through the FreeBSD installation process and gently introduces the concepts
and conventions that underpin UNIX®.
Working through this section requires little more than the desire to explore,
and the ability to take on board new concepts as they are introduced.

Once you have traveled this far, the second, far larger, section of the Handbook
is a comprehensive reference to all manner of topics of interest to FreeBSD
system administrators.
Some of these chapters may recommend that you do some prior reading, and this is
noted in the synopsis at the beginning of each chapter.

For a list of additional sources of information, please see Appendix B,
\emph{Bibliography}.




\section{Changes from the Third Edition}

The current online version of the Handbook represents the cumulative effort of
many hundreds of contributors over the past 10 years.
The following are some of the significant changes since the two volume third
edition was published in 2004:
\begin{itemize}
\item
   Chapter 24, \emph{DTrace}, has been added with information about the powerful
   DTrace performance analysis tool.
\item
   Chapter 20, \emph{Other File Systems}, has been added with information about
   non-native file systems in FreeBSD, such as ZFS from Sun™.
\item
   Chapter 16, \emph{Security Event Auditing}, has been added to cover the new
   auditing capabilities in FreeBSD and explain its use.
\item
   Chapter 21, \emph{Virtualization}, has been added with information about
   installing FreeBSD on virtualization software.
\item
   \Cref{chap:installing-freebsd}, \nameref{chap:installing-freebsd} \emph{Installing FreeBSD}, has been added to cover installation of
   FreeBSD using the new installation utility, \cmd{bsdinstall}.
\end{itemize}



\section{Changes from the Second Edition (2004)}

The third edition was the culmination of over two years of work by the dedicated
members of the FreeBSD Documentation Project.
The printed edition grew to such a size that it was necessary to publish as two
separate volumes.
The following are the major changes in this new edition:
\begin{itemize}
\item
   Chapter 11, \emph{Configuration and Tuning}, has been expanded with new
   information about the ACPI power and resource management, the \cmd{cron}
   system utility, and more kernel tuning options.
\item
   Chapter 13, \emph{Security}, has been expanded with new information about
   virtual private networks (VPNs), file system access control lists (ACLs), and
   security advisories.
\item
   Chapter 15, \emph{Mandatory access control}, is a new chapter with this
   edition.
   It explains what MAC is and how this mechanism can be used to secure a
   FreeBSD system.
\item
   Chapter 17, \emph{Storage}, has been expanded with new information about USB
   storage devices, file system snapshots, file system quotas, file and network
   backed filesystems, and encrypted disk partitions.
\item
   A troubleshooting section has been added to Chapter 27, \emph{PPP}.
\item
   Chapter 28, \emph{Electronic Mail}, has been expanded with new information
   about using alternative transport agents, SMTP authentication, UUCP,
   fetchmail, procmail, and other advanced topics.
\item
   Chapter 29, \emph{Network Servers}, is all new with this edition.
   This chapter includes information about setting up the Apache HTTP Server,
   ftpd, and setting up a server for Microsoft® Windows® clients with Samba.
   Some sections from Chapter 31, \emph{Advanced Networking}, were moved here to
   improve the presentation.
\item
   Chapter 31, \emph{Advanced Networking}, has been expanded with new
   information about using Bluetooth® devices with FreeBSD, setting up wireless
   networks, and Asynchronous Transfer Mode (ATM) networking.
\item
   A glossary has been added to provide a central location for the definitions
   of technical terms used throughout the book.
\item
   A number of aesthetic improvements have been made to the tables and figures
   throughout the book.
\end{itemize}



\section{Changes from the First Edition (2001)}

The second edition was the culmination of over two years of work by the
dedicated members of the FreeBSD Documentation Project.
The following were the major changes in this edition:
\begin{itemize}
\item
   A complete Index has been added.
\item
   All ASCII figures have been replaced by graphical diagrams.
\item
   A standard synopsis has been added to each chapter to give a quick summary of
   what information the chapter contains, and what the reader is expected to
   know.
\item
   The content has been logically reorganized into three parts:
   ``Getting Started,'' ``System Administration,'' and ``Appendices.''
\item
   Chapter 3, \emph{FreeBSD Basics}, has been expanded to contain additional
   information about processes, daemons, and signals.
\item
   Chapter 4, \emph{Installing Applications: Packages and Ports}, has been
   expanded to contain additional information about binary package management.
\item
   Chapter 5, \emph{The X Window System}, has been completely rewritten with an
   emphasis on using modern desktop technologies such as KDE and GNOME on
   XFree86™ 4.X.
\item
   Chapter 12, \emph{The FreeBSD Booting Process} has been expanded.
\item
   Chapter 17, \emph{Storage} has been written from what used to be two separate
   chapters on ``Disks'' and ``Backups.''
   We feel that the topics are easier to comprehend when presented as a single
   chapter.
   A section on RAID (both hardware and software) has also been added.
\item
   Chapter 26, \emph{Serial Communications} has been completely reorganized and
   updated for FreeBSD 4.X/5.X.
\item
   Chapter 27, \emph{PPP}, has been substantially updated.
\item
   Many new sections have been added to Chapter 31, \emph{Advanced Networking}.
\item
   Chapter 28, \emph{Electronic Mail} has been expanded to include more
   information about configuring sendmail.
\item
   Chapter 10, \emph{Linux® Binary Compatibility} has been expanded to include
   information about installing Oracle® and SAP® R/3®.
\item
   The following new topics are covered in this second edition:
   \begin{itemize}
   \item
      Chapter 11, \emph{Configuration and Tuning}, and
   \item
      Chapter 7, \emph{Multimedia}.
   \end{itemize}
\end{itemize}




\section{Organization of this Book}

This book is split into five logically distinct sections.
The first section, \emph{Getting Started}, covers the installation and basic
usage of FreeBSD.
It is expected that the reader will follow these chapters in sequence, possibly
skipping chapters covering familiar topics.
The second section, \emph{Common Tasks}, covers some frequently used features of
FreeBSD.
This section, and all subsequent sections, can be read out of order.
Each chapter begins with a succinct synopsis that describes what the chapter
covers and what the reader is expected to already know.
This is meant to allow the casual reader to skip around to find chapters of
interest.
The third section, \emph{System Administration}, covers administration topics.
The fourth section, \emph{Network Communication}, covers networking and server
topics.
The fifth section contains appendices of reference information.















\section{Conventions Used in this Book}

To provide a consistent and easy to read text, several conventions are followed
throughout the book.

\subsection{Typographic conventions}

\begin{description}
\item[italic]
   An \emph{italic} font is used for filenames, URLs, emphasized text, and the
   first usage of technical terms.
\item[monospace]
   A \texttt{monospaced} font is used for error messages, commands, environment
   variables, names of ports, hostnames, user names, group names, device names,
   variables, and code fragments.
\item[bold]
   A \textbf{bold} font is used for applications, commands, and keys.
\end{description}


\subsection{User Input}

Keys are shown in \textbf{bold} to stand out from other text.
Key combinations that are meant to be typed simultaneously are shown with `+'
between the keys, such as: \key{Ctrl+Alt+Del}, meaning the user should type
the \key{Ctrl}, \key{Alt}, and \key{Del} keys at the same time.
Keys that are meant to be typed in sequence will be separated with commas, for
example: \key{Ctrl+X}, \key{Ctrl+S} would mean that the user is expected to type
the \key{Ctrl} and \key{X} keys simultaneously and then to type the \key{Ctrl}
and \key{S} keys simultaneously.


\subsection{Examples}

Examples starting with \texttt{C:\textbackslash\textgreater} indicate an MS-DOS® command.
Unless otherwise noted, these commands may be executed from a ``Command Prompt''
window in a modern Microsoft® Windows® environment.
\begin{lstlisting}
E:\> tools\fdimage floppies\kern.flp A:
\end{lstlisting}

Examples starting with \texttt{\#} indicate a command that must be invoked as the
superuser in FreeBSD.
You can login as \user{root} to type the command, or login as your normal
account and use \man[1]{su} to gain superuser privileges.
\begin{lstlisting}
# dd if=kern.flp of=/dev/fd0
\end{lstlisting}

Examples starting with \texttt{\%} indicate a command that should be invoked
from a normal user account.
Unless otherwise noted, C-shell syntax is used for setting environment variables
and other shell commands.
\begin{lstlisting}
% top
\end{lstlisting}





\section{Acknowledgements}

The book you are holding represents the efforts of many hundreds of people
around the world.
Whether they sent in fixes for typos, or submitted complete chapters, all the
contributions have been useful.

Several companies have supported the development of this document by paying
authors to work on it full-time, paying for publication, etc.
In particular, BSDi (subsequently acquired by Wind River Systems) paid members
of the FreeBSD Documentation Project to work on improving this book full time
leading up to the publication of the first printed edition in March 2000
(ISBN 1-57176-241-8).
Wind River Systems then paid several additional authors to make a number of
improvements to the print-output infrastructure and to add additional chapters
to the text.
This work culminated in the publication of the second printed edition in
November 2001 (ISBN 1-57176-303-1).
In 2003-2004, FreeBSD Mall, Inc, paid several contributors to improve the
Handbook in preparation for the third printed edition.