\chapter{Installing FreeBSD}
\label{chap:installing-freebsd}
\chapterprecis{%
   Restructured, reorganized, and parts rewritten by Jim Mock.
   Updated for \cmd{bsdinstall} by Gavin Atkinson and Warren Block.
   Updated for root-on-ZFS by Allan Jude.
}

\section{Synopsis}

There are several different ways of getting FreeBSD to run, depending on the environment.
These are:
\begin{itemize}
\item
   virtual machine images, to download and import on a virtual environment of choice.
   These can be downloaded from the Download FreeBSD page.
   There are images for KVM (``qcow2''), VMWare (``vmdk''), Hyper-V (``vhd''), and raw device images that are universally supported.
    These are not installation images, but rather the preconfigured (``already installed'') instances, ready to run and perform post-installation tasks.
\item
   virtual machine images available at Amazon's AWS Marketplace, Microsoft Azure Marketplace, and Google Cloud Platform, to run on their respective hosting services.
   For more information on deploying FreeBSD on Azure please consult the relevant chapter in the Azure Documentation.
\item
   SD card images, for embedded systems such as Raspberry Pi or BeagleBone Black.
   These can be downloaded from the Download FreeBSD page.
   These files must be uncompressed and written as a raw image to an SD card, from which the board will then boot.
\item
   installation images, to install FreeBSD on a hard drive for the usual desktop, laptop, or server systems.
\end{itemize}
The rest of this chapter describes the fourth case, explaining how to install FreeBSD using the text-based installation program named \cmd{bsdinstall}.

In general, the installation instructions in this chapter are written for the i386™ and AMD64 architectures.
Where applicable, instructions specific to other platforms will be listed.
There may be minor differences between the installer and what is shown here, so use this chapter as a general guide rather than as a set of literal instructions.

\begin{note}
Users who prefer to install FreeBSD using a graphical installer may be interested in FuryBSD, GhostBSD or MidnightBSD.
\end{note}

After reading this chapter, you will know:
\begin{itemize}
\item
   the minimum hardware requirements and FreeBSD supported architectures,
\item
   how to create the FreeBSD installation media,
\item
   how to start \cmd{bsdinstall},
\item
   the questions \cmd{bsdinstall} will ask, what they mean, and how to answer them,
\item
   how to troubleshoot a failed installation,
\item
   how to access a live version of FreeBSD before committing to an installation.
\end{itemize}
Before reading this chapter, you should:
\begin{itemize}
\item
   read the supported hardware list that shipped with the version of FreeBSD to be installed and verify that the system's hardware is supported.
\end{itemize}



\section{Minimum Hardware Requirements}

The hardware requirements to install FreeBSD vary by architecture.
Hardware architectures and devices supported by a FreeBSD release are listed on the FreeBSD Release Information page.
The FreeBSD download page also has recommendations for choosing the correct image for different architectures.

A FreeBSD installation requires a minimum of 96\,MB of RAM and 1.5\,GB of free hard drive space.
However, such small amounts of memory and disk space are really only suitable for custom applications like embedded appliances.
General-purpose desktop systems need more resources, 2--4\,GB RAM and at least 8\,GB hard drive space is a good starting point.

These are the processor requirements for each architecture.
\begin{description}
\item[amd64]
   This is the most common desktop and laptop processor type, used in most modern systems.
   Intel® calls it Intel64.
   Other manufacturers sometimes call it x86-64.

   Examples of amd64 compatible processors include: AMD Athlon™64, AMD Opteron™, multi-core Intel® Xeon™, and Intel® Core™ 2 and later processors.
\item[i386]
   Older desktops and laptops often use this 32-bit, x86 architecture.

   Almost all i386-compatible processors with a floating point unit are supported.
   All Intel® processors 486 or higher are supported.

   FreeBSD will take advantage of Physical Address Extensions (PAE) support on CPUs with this feature.
   A kernel with the PAE feature enabled will detect memory above 4\,GB and allow it to be used by the system.
   However, using PAE places constraints on device drivers and other features of FreeBSD.
\item[powerpc]
   All New World ROM Apple® Mac® systems with built-in USB are supported.
   SMP is supported on machines with multiple CPUs.

   A 32-bit kernel can only use the first 2 GB of RAM.
\item[sparc64]
   Systems supported by FreeBSD/sparc64 are listed at the FreeBSD/sparc64 Project.

   SMP is supported on all systems with more than 1 processor.
   A dedicated disk is required as it is not possible to share a disk with another operating system at this time.
\end{description}




\section{Pre-Installation Tasks}
Once it has been determined that the system meets the minimum hardware requirements for installing FreeBSD, the installation file should be downloaded and the installation media prepared.
Before doing this, check that the system is ready for an installation by verifying the items in this checklist.

\begin{enumerate}
\item
   Back up important data

   Before installing any operating system, always backup all important data first.
   Do not store the backup on the system being installed.
   Instead, save the data to a removable disk such as a USB drive, another system on the network, or an online backup service.
   Test the backup before starting the installation to make sure it contains all of the needed files.
   Once the installer formats the system's disk, all data stored on that disk will be lost.
\item 
   Decide where to install FreeBSD

   If FreeBSD will be the only operating system installed, this step can be skipped.
   But if FreeBSD will share the disk with another operating system, decide which disk or partition will be used for FreeBSD.

   In the i386 and amd64 architectures, disks can be divided into multiple partitions using one of two partitioning schemes.
   A traditional \termi[MBR]{Master Boot Record} holds a partition table defining up to four primary \termi{partitions}.
   For historical reasons, FreeBSD calls these primary partition \termi{slices}.
   One of these primary partitions can be made into an \term{extended partition} containing multiple \term{logical partitions}.
   The \termi[GPT]{GUID Partition Table} is a newer and simpler method of partitioning a disk.
   Common GPT implementations allow up to 128 partitions per disk, eliminating the need for logical partitions.

   The FreeBSD boot loader requires either a primary or GPT partition.
   If all of the primary or GPT partitions are already in use, one must be freed for FreeBSD.
   To create a partition without deleting existing data, use a partition resizing tool to shrink an existing partition and create a new partition using the freed space.

   A variety of free and commercial partition resizing tools are listed at \url{http://en.wikipedia.org/wiki/List_of_disk_partitioning_software}.
   GParted Live (\url{http://gparted.sourceforge.net/livecd.php}) is a free live CD which includes the GParted partition editor.
   GParted is also included with many other Linux live CD distributions.

   \begin{warning}
   When used properly, disk shrinking utilities can safely create space for creating a new partition.
   Since the possibility of selecting the wrong partition exists, always backup any important data and verify the integrity of the backup before modifying disk partitions.
   \end{warning}

   Disk partitions containing different operating systems make it possible to install multiple operating systems on one computer.
   An alternative is to use virtualization (Chapter 21, \emph{Virtualization}) which allows multiple operating systems to run at the same time without modifying any disk partitions.
\item    
   Collect network information

   Some FreeBSD installation methods require a network connection in order to download the installation files.
   After any installation, the installer will offer to setup the system's network interfaces.

   If the network has a DHCP server, it can be used to provide automatic network configuration.
   If DHCP is not available, the following network information for the system must be obtained from the local network administrator or Internet service provider:
   \begin{itemize}
   \item IP address,
   \item subnet mask,
   \item IP address of the default gateway,
   \item domain name of the network,
   \item IP addresses of the network's DNS servers.
   \end{itemize}
\item
   Check for FreeBSD errata

   Although the FreeBSD Project strives to ensure that each release of FreeBSD is as stable as possible, bugs occasionally creep into the process.
   On very rare occasions those bugs affect the installation process.
   As these problems are discovered and fixed, they are noted in the FreeBSD Errata (\url{https://www.freebsd.org/releases/12.1R/errata.html}) on the FreeBSD web site.
   Check the errata before installing to make sure that there are no problems that might affect the installation.
\end{enumerate}



\subsection{Prepare the Installation Media}

The FreeBSD installer is not an application that can be run from within another operating system.
Instead, download a FreeBSD installation file, burn it to the media associated with its file type and size (CD, DVD, or USB), and boot the system to install from the inserted media.

FreeBSD installation files are available at \url{www.freebsd.org/where.html#download}.
Each installation file's name includes the release version of FreeBSD, the architecture, and the type of file.
For example, to install FreeBSD 12.1 on an amd64 system from a DVD, download \file{FreeBSD-12.1-RELEASE-amd64-dvd1.iso}, burn this file to a DVD, and boot the system with the DVD inserted.

Installation files are available in several formats.
The formats vary depending on computer architecture and media type.

Additional installation files are included for computers that boot with UEFI (Unified Extensible Firmware Interface).
The names of these files include the string ``uefi.''

File types:
\begin{itemize}
\item
   \file{-bootonly.iso}:
   This is the smallest installation file as it only contains the installer.
   A working Internet connection is required during installation 
   as the installer will download the files it needs to complete the FreeBSD installation.
   This file should be burned to a CD using a CD burning application.
\item 
   \file{-disc1.iso}:
   This file contains all of the files needed to install FreeBSD, its source, and the Ports Collection.
   It should be burned to a CD using a CD burning application.
\item
   \file{-dvd1.iso}:
   This file contains all of the files needed to install FreeBSD, its source, and the Ports Collection.
   It also contains a set of popular binary packages for installing a window manager and some applications so that a complete system can be installed from media without requiring a connection to the Internet. 
   This file should be burned to a DVD using a DVD burning application.
\item
   \file{-memstick.img}:
   This file contains all of the files needed to install FreeBSD, its source, and the Ports Collection.
   It should be burned to a USB stick using the instructions below.
\item
   \file{-mini-memstick.img}:
   Like \file{-bootonly.iso}, does not include installation files, but downloads them as needed.
   A working internet connection is required during installation.
   Write this file to a USB stick as shown in \vref{sec:writing-image-to-usb}.
\end{itemize}
After downloading the image file, download \file{CHECKSUM.SHA256} from the same directory.
Calculate a checksum for the image file.
FreeBSD provides \man[1]{sha256} for this, used as \cmd{sha256 \param{imagefilename}}.
Other operating systems have similar programs.

Compare the calculated checksum with the one shown in \file{CHECKSUM.SHA256}.
The checksums must match exactly.
If the checksums do not match, the image file is corrupt and must be downloaded again.

\subsubsection{Writing an Image File to a USB Disk}
\label{sec:writing-image-to-usb}

The \file{*.img} file is an image of the complete contents of a memory stick.
It cannot be copied to the target device as a file.
Several applications are available for writing the \file{*.img} to a USB stick.
This section describes two of these utilities.

\begin{important}
Before proceeding, back up any important data on the USB stick.
This procedure will erase the existing data on the stick.
\end{important}

\paragraph{Using \cmd{dd} to Write the Image}
\begin{warning}
This example uses \file{/dev/da0} as the target device where the image will be written.
Be \emph{very} careful that the correct device is used as this command will destroy the existing data on the specified target device.
\end{warning}

The \man[1]{dd} command-line utility is available on BSD, Linux®, and Mac OS® systems.
To burn the image using \cmd{dd}, insert the USB stick and determine its device name.
Then, specify the name of the downloaded installation file and the device name for the USB stick.
This example burns the amd64 installation image to the first USB device on an existing FreeBSD system.
\begin{lstlisting}
# dd if=FreeBSD-12.1-RELEASE-amd64-memstick.img of=/dev/da0 bs=1M conv=sync
\end{lstlisting}
If this command fails, verify that the USB stick is not mounted and that the device name is for the disk, not a partition.
Some operating systems might require this command to be run with \man[8]{sudo}.
The \man[1]{dd} syntax varies slightly across different platforms; for example, Mac OS® requires a lower-case ``bs=1m.''
Systems like Linux® might buffer writes.
To force all writes to complete, use \man[8]{sync}.

\paragraph{Using Windows® to Write the Image}
\begin{warning}
Be sure to give the correct drive letter as the existing data on the specified drive will be overwritten and destroyed.
\end{warning}
\begin{enumerate}
\item
   Obtaining Image Writer for Windows®

   Image Writer for Windows® is a free application that can correctly write an image file to a memory stick.
   Download it from \url{https://sourceforge.net/projects/win32diskimager/} and extract it into a folder.
\item
   Writing the Image with Image Writer

   Double-click the \cmd{Win32DiskImager} icon to start the program. 
   Verify that the drive letter shown under Device is the drive with the memory stick.
   Click the folder icon and select the image to be written to the memory stick.
   Click \key{[ Save ]} to accept the image file name.
   Verify that everything is correct, and that no folders on the memory stick are open in other windows.
   When everything is ready, click \key{[ Write ]} to write the image file to the memory stick.
\end{enumerate}
You are now ready to start installing FreeBSD.





\section{Starting the Installation}
